\appendix
\section{Epistemic Metrics}
\label{appendix:epistemic_metrics}

We introduce \emph{epistemic metrics} to quantify how the system acquires, tests, revises, and transfers knowledge across experimental blocks. These metrics characterize system-level behaviors such as hypothesis refinement, diagnostic experimentation, regime differentiation, and the efficient transfer of previously established principles when entering new regions of the search space. Importantly, these metrics do not measure task performance alone, but the structure and dynamics of the exploration process itself.

\subsection{Reasoning Mode Definitions}

We classify reasoning events into twelve epistemic modes, each capturing a distinct cognitive operation:

\begin{description}
    \item[Induction] Pattern extraction from accumulated observations. Measured by the number of observations required before a generalizable pattern is articulated.
    \item[Abduction] Generation of causal hypotheses to explain unexpected observations. Typically triggered by performance anomalies or regime changes.
    \item[Deduction] Derivation of testable predictions from hypotheses. Enables hypothesis validation through targeted experiments.
    \item[Falsification] Rejection of hypotheses based on contradictory evidence. Critical for pruning the hypothesis space and refining understanding.
    \item[Boundary] Systematic probing of parameter limits to establish operating ranges. Identifies safe vs.\ failure regions.
    \item[Analogy/Transfer] Application of knowledge from one regime to another. Success rate indicates generalizability of learned principles.
    \item[Meta-reasoning] Reflection on and adaptation of exploration strategy. Triggered when current approaches prove ineffective.
    \item[Regime] Identification of qualitatively distinct operating phases. Enables context-appropriate hypothesis generation.
    \item[Uncertainty] Recognition of stochastic or irreducible variability. Informs confidence bounds and exploration intensity.
    \item[Causal] Construction of multi-step mechanistic chains linking variables. Represents deep structural understanding.
    \item[Predictive] Formulation of quantitative rules enabling \emph{a priori} predictions. Highest form of operational knowledge.
    \item[Constraint] Derivation of parameter relationships that must hold for success. Enables principled configuration.
\end{description}

\subsection{Primary Metrics}

\subsubsection{Hypothesis Turnover Rate (HTR)}

The rate at which hypotheses are generated and subsequently falsified or refined:
\begin{equation}
    \text{HTR} = \frac{N_{\text{falsified}} + N_{\text{refined}}}{N_{\text{abduction}}}
\end{equation}
where $N_{\text{falsified}}$ is the count of falsification events, $N_{\text{refined}}$ the count of hypothesis refinements, and $N_{\text{abduction}}$ the count of abductive hypotheses generated. High HTR indicates active hypothesis testing; HTR $\approx 1$ suggests efficient single-shot hypothesis validation; HTR $> 1$ indicates iterative refinement cycles.

\subsubsection{Deductive Accuracy (DA)}

The fraction of deductive predictions confirmed by subsequent experiments:
\begin{equation}
    \text{DA} = \frac{N_{\text{deduction}}^{\checkmark}}{N_{\text{deduction}}^{\checkmark} + N_{\text{deduction}}^{\times}}
\end{equation}
where $N_{\text{deduction}}^{\checkmark}$ counts confirmed predictions and $N_{\text{deduction}}^{\times}$ counts falsified predictions. DA reflects the quality of the underlying hypotheses and the system's predictive calibration.

\subsubsection{Transfer Success Rate (TSR)}

The effectiveness of cross-regime knowledge transfer:
\begin{equation}
    \text{TSR} = \frac{N_{\text{transfer}}^{\text{success}} + 0.5 \cdot N_{\text{transfer}}^{\text{partial}}}{N_{\text{transfer}}^{\text{total}}}
\end{equation}
where transfer attempts are classified as success ($\checkmark$), partial ($\sim$), or failure ($\times$). TSR quantifies the generalizability of learned principles across different experimental regimes.

\subsubsection{Causal Depth (CD)}

The average length of mechanistic causal chains constructed:
\begin{equation}
    \text{CD} = \frac{1}{N_{\text{causal}}} \sum_{i=1}^{N_{\text{causal}}} L_i
\end{equation}
where $L_i$ is the number of variables in causal chain $i$. Higher CD indicates deeper mechanistic understanding beyond surface correlations.

\subsubsection{Epistemic Density (ED)}

The concentration of reasoning events per iteration:
\begin{equation}
    \text{ED} = \frac{N_{\text{events}}}{N_{\text{iterations}}}
\end{equation}
where $N_{\text{events}}$ is the total count of classified reasoning events. ED captures the intensity of epistemic activity; higher values indicate richer reasoning per experimental step.

\subsection{Structural Metrics}

\subsubsection{Falsification Latency (FL)}

The average number of iterations between hypothesis generation and falsification:
\begin{equation}
    \text{FL} = \frac{1}{N_{\text{falsified}}} \sum_{j=1}^{N_{\text{falsified}}} (t_j^{\text{falsify}} - t_j^{\text{hypothesis}})
\end{equation}
Short FL indicates rapid hypothesis testing; long FL may indicate either robust hypotheses or insufficient testing.

\subsubsection{Induction Consolidation Time (ICT)}

The number of observations required before pattern articulation:
\begin{equation}
    \text{ICT} = \frac{1}{N_{\text{induction}}} \sum_{k=1}^{N_{\text{induction}}} n_k^{\text{obs}}
\end{equation}
where $n_k^{\text{obs}}$ is the observation count preceding induction event $k$. Lower ICT indicates faster pattern recognition.

\subsubsection{Regime Sensitivity Index (RSI)}

The fraction of block transitions that trigger regime recognition:
\begin{equation}
    \text{RSI} = \frac{N_{\text{regime}}}{N_{\text{blocks}} - 1}
\end{equation}
RSI $= 1$ indicates perfect regime awareness; lower values suggest failure to recognize qualitative changes in the problem structure.

\subsubsection{Meta-Reasoning Trigger Rate (MTR)}

The frequency of strategy-level adaptation:
\begin{equation}
    \text{MTR} = \frac{N_{\text{meta}}}{N_{\text{blocks}}}
\end{equation}
Higher MTR indicates more frequent strategic reflection, typically triggered by persistent failures or unexpected successes.

\subsection{Graph-Theoretic Metrics}

The epistemic timeline can be represented as a directed graph $G = (V, E)$ where vertices $V$ are reasoning events and edges $E$ represent causal or temporal dependencies.

\subsubsection{Reasoning Chain Length (RCL)}

The average path length in the epistemic graph:
\begin{equation}
    \text{RCL} = \frac{1}{|P|} \sum_{p \in P} |p|
\end{equation}
where $P$ is the set of maximal directed paths and $|p|$ is path length. Longer chains indicate sustained reasoning threads.

\subsubsection{Cross-Block Connectivity (CBC)}

The ratio of cross-block to within-block edges:
\begin{equation}
    \text{CBC} = \frac{|E_{\text{cross}}|}{|E_{\text{within}}|}
\end{equation}
Higher CBC indicates stronger integration of knowledge across experimental phases.

\subsubsection{Falsification Fan-In (FFI)}

The average number of hypotheses rejected by each falsification event:
\begin{equation}
    \text{FFI} = \frac{|\{e \in E : \text{target}(e) \in V_{\text{falsification}}\}|}{|V_{\text{falsification}}|}
\end{equation}
FFI $> 1$ indicates efficient falsification that eliminates multiple competing hypotheses simultaneously.

\subsection{Temporal Dynamics}

\subsubsection{Epistemic Phase Transition}

We identify phase transitions in reasoning complexity by tracking the emergence of advanced modes (Causal, Predictive, Constraint) over time. Define the \emph{epistemic maturity} at iteration $t$:
\begin{equation}
    M(t) = \frac{N_{\text{causal}}(t) + N_{\text{predictive}}(t) + N_{\text{constraint}}(t)}{N_{\text{events}}(t)}
\end{equation}
The first iteration where $M(t) > 0$ marks the transition from exploratory to explanatory reasoning.

\subsubsection{Knowledge Crystallization Rate (KCR)}

The rate at which predictive rules emerge from accumulated experience:
\begin{equation}
    \text{KCR} = \frac{N_{\text{predictive}}}{N_{\text{iterations}}} \times 100
\end{equation}
expressed as predictive rules per 100 iterations. Higher KCR indicates efficient distillation of experience into actionable knowledge.

\subsection{Computed Metrics for signal\_chaotic\_1\_Claude}

Table~\ref{tab:epistemic_metrics} presents the epistemic metrics computed from 149 iterations across 11 experimental blocks.

\begin{table}[htbp]
\centering
\caption{Epistemic metrics for the signal\_chaotic\_1\_Claude experiment.}
\label{tab:epistemic_metrics}
\begin{tabular}{lrp{6cm}}
\toprule
\textbf{Metric} & \textbf{Value} & \textbf{Interpretation} \\
\midrule
\multicolumn{3}{l}{\emph{Primary Metrics}} \\
Hypothesis Turnover Rate (HTR) & 2.43 & Active iterative refinement \\
Deductive Accuracy (DA) & 0.71 & 71\% of predictions confirmed \\
Transfer Success Rate (TSR) & 0.35 & Limited cross-regime generalization \\
Causal Depth (CD) & 3.6 & Multi-step mechanistic chains \\
Epistemic Density (ED) & 0.72 & 0.72 reasoning events per iteration \\
\midrule
\multicolumn{3}{l}{\emph{Structural Metrics}} \\
Falsification Latency (FL) & 4.2 iter & Rapid hypothesis testing \\
Induction Consolidation Time (ICT) & 6.8 obs & Moderate pattern recognition speed \\
Regime Sensitivity Index (RSI) & 0.70 & 7/10 block transitions recognized \\
Meta-Reasoning Trigger Rate (MTR) & 0.55 & Strategy adaptation in 6/11 blocks \\
\midrule
\multicolumn{3}{l}{\emph{Graph-Theoretic Metrics}} \\
Total Edges & 53 & Causal relationships identified \\
Within-Block Edges & 44 & Local reasoning chains \\
Cross-Block Edges & 9 & Knowledge transfer links \\
Cross-Block Connectivity (CBC) & 0.20 & Moderate integration \\
\midrule
\multicolumn{3}{l}{\emph{Temporal Dynamics}} \\
First Causal Chain & Iter 48 & Epistemic phase transition \\
Knowledge Crystallization Rate (KCR) & 2.01 & 3 predictive rules / 149 iterations \\
\bottomrule
\end{tabular}
\end{table}

\subsection{Mode Distribution}

The distribution of reasoning events by mode reveals the epistemic profile of the exploration:

\begin{table}[htbp]
\centering
\caption{Distribution of reasoning events by epistemic mode.}
\label{tab:mode_distribution}
\begin{tabular}{lrrp{5cm}}
\toprule
\textbf{Mode} & \textbf{Count} & \textbf{\%} & \textbf{Role} \\
\midrule
Falsification & 20 & 18.5\% & Hypothesis elimination \\
Deduction & 17 & 15.7\% & Prediction generation \\
Analogy/Transfer & 10 & 9.3\% & Cross-regime application \\
Boundary & 12 & 11.1\% & Parameter limit probing \\
Induction & 13 & 12.0\% & Pattern recognition \\
Abduction & 10 & 9.3\% & Hypothesis generation \\
Regime & 7 & 6.5\% & Phase identification \\
Meta-reasoning & 6 & 5.6\% & Strategy adaptation \\
Uncertainty & 5 & 4.6\% & Stochasticity awareness \\
Causal & 5 & 4.6\% & Mechanistic modeling \\
Predictive & 3 & 2.8\% & Quantitative rules \\
Constraint & 1 & 0.9\% & Parameter relationships \\
\midrule
\textbf{Total} & \textbf{108} & \textbf{100\%} & \\
\bottomrule
\end{tabular}
\end{table}

\subsection{Interpretation}

The epistemic metrics reveal several key characteristics of the LLM-driven exploration:

\paragraph{Hypothesis-Driven Exploration.} The high HTR (2.43) indicates that hypotheses undergo multiple refinement cycles rather than single-shot validation. Combined with DA = 0.71, this suggests calibrated but not overconfident prediction.

\paragraph{Limited Transfer.} TSR = 0.35 reflects the challenge of cross-regime generalization. Most knowledge learned in one block required adaptation when applied to structurally different regimes (e.g., Dale's law, low-rank constraints).

\paragraph{Emergent Causal Understanding.} The appearance of causal chains at iteration 48 marks a phase transition from parameter search to mechanistic modeling. The average causal depth of 3.6 indicates multi-step reasoning beyond pairwise correlations.

\paragraph{Balanced Reasoning Profile.} The mode distribution shows a roughly balanced profile between constructive (Induction, Abduction, Deduction) and critical (Falsification, Boundary) reasoning, with meta-cognitive modes (Meta-reasoning, Regime, Uncertainty) providing adaptive control.

\paragraph{Knowledge Crystallization.} Despite 149 iterations, only 3 predictive rules emerged (KCR = 2.01 per 100 iterations), suggesting that converting experimental experience into generalizable quantitative principles remains challenging.
