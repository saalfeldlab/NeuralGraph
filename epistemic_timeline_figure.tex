\section{Results}

\begin{figure*}[htbp]
\centering
\includegraphics[width=\textwidth]{signal_chaotic_1_Claude_epistemic_timeline.png}
\caption{\textbf{Epistemic reasoning timeline across 149 iterations (11 blocks).} Each node represents a reasoning event, colored by mode and sized by significance. Blocks alternate white/gray backgrounds. Edges show causal relationships: solid gray (leads\_to), dashed blue (triggers), red backward arrows (to Falsification). \textbf{Modes:} \emph{Induction}: pattern extraction from observations; \emph{Boundary}: parameter limit testing; \emph{Abduction}: causal hypothesis generation; \emph{Deduction}: prediction from hypothesis; \emph{Falsification}: hypothesis rejection; \emph{Analogy/Transfer}: cross-regime knowledge transfer; \emph{Meta-reasoning}: strategy adaptation; \emph{Regime}: phase identification; \emph{Uncertainty}: stochasticity awareness; \emph{Constraint}: parameter constraint formulation; \emph{Predictive}: quantitative rule derivation; \emph{Causal}: mechanistic chain modeling.}
\label{fig:epistemic_timeline}
\end{figure*}

There is a pyramidal reasoning scheme. At the bottom, inductions and abductions occur throughout the process, forming the foundation of hypothesis generation and pattern recognition. This layer supports a second layer of deductions---testable predictions derived from hypotheses. All of these reasoning events can be falsified or refined, as shown by the red nodes appearing after blue predictions. Ultimately, causal chains of reasoning emerge. The first appeared at iteration 48, where the chain $\texttt{connectivity\_rank} \to \texttt{eff\_rank} \to R^2_{\text{ceiling}} \to \texttt{convergence}$ was established when the LLM constructed a multi-step mechanistic model explaining why low-rank connectivity limits GNN recovery. A second causal node at iteration 127 formalized the $\texttt{low\_rank=10} \to \texttt{eff\_rank=4} \to \texttt{unlearnable}$ barrier. At iteration 135, Dale's law stabilization ($\rho: 1.0 \to 0.65$) was causally linked to increased effective rank. The predictive nodes (light green, iterations 88, 96, 159) represent quantitative models: the $\texttt{eff\_rank} \geq 20 \to R^2 \approx 0.999$ threshold and the $2\times$ overparameterization principle. The constraint node (orange, iteration 104) captures the batch\_size$\to$lr\_W compensation rule. Cross-block edges (dashed blue spanning blocks) show knowledge transfer, with block summaries triggering analogical reasoning in subsequent regimes.

To quantify these epistemic behaviors beyond simple mode counts, we introduce \emph{epistemic metrics} (Appendix~\ref{appendix:epistemic_metrics}). Key findings include: a Hypothesis Turnover Rate of 2.43 indicating iterative refinement cycles; Deductive Accuracy of 71\% reflecting calibrated prediction; Transfer Success Rate of only 35\% highlighting the challenge of cross-regime generalization; and a Causal Depth of 3.6 demonstrating multi-step mechanistic reasoning. The first causal chain emerged at iteration 48, marking the transition from parameter search to explanatory modeling.

\section{Discussion}
